% !TEX encoding = UTF-8 Unicode
% -*- coding: UTF-8; -*-
% vim: set fenc=utf-8
\documentclass[svgnames]{beamer}
\setlength{\tabcolsep}{3em}
\usetheme[pageofpages=of,% String used between the current page and the
                         % total page count.
          titleline=true,
          alternativetitlepage=true% Use the fancy title page.
          ]{Torino}
\usecolortheme{metascale}
\usepackage{listings}

\ifxetex
  \usepackage{fontspec}
  \defaultfontfeatures{Mapping=tex-text}
  \setsansfont[ItalicFont={GillSansMTPro-BookItalic}]{GillSansMTPro-Book}
  \setmonofont{Inconsolata}
  \newcommand{\codefont}{\ttfamily}
\else
  \usepackage[utf8x]{inputenc}
  %\usepackage[nott]{inconsolata}
  %\newcommand{\codefont}{\fontfamily{fi4}\selectfont}
  \usepackage{inconsolata}
  \newcommand{\codefont}{\ttfamily}
\fi
\usepackage{graphicx}
\usepackage{color}
%\usepackage{multicol}
%\usepackage{array}
%\usepackage{colortbl}
\usepackage[frenchb]{babel}
%\usepackage{pgfplots}
%\usepackage{standalone}
\usepackage{relsize}

% setup tikz
%% \usepackage{tikz}
%% \usetikzlibrary{calc,trees,positioning,arrows,chains,shapes.geometric,%
%% decorations.pathreplacing,decorations.pathmorphing,shapes,%
%% matrix,shapes.symbols,plotmarks,decorations.markings,shadows,%
%% snakes,backgrounds}
\def\print#1{\pgfmathparse{#1}\pgfmathresult}
\def\colx{Blue!40}
\def\coly{Blue!20}
\def\colz{white}

\def\C++{\textup{C}\nolinebreak[4]\hspace{-.05em}\raisebox{.4ex}{\relsize{-3}{\textbf{++}}}}

% vertical align box
\newcommand*{\vcenteredhbox}[1]{\begingroup
\setbox0=\hbox{#1}\parbox{\wd0}{\box0}\endgroup}

\lstdefinestyle{customcpp}{language=C++,
        basicstyle=\footnotesize\codefont,
        tabsize=2,
        numberstyle=\footnotesize,
        showstringspaces=false,
        %columns=fullflexible,
        keywordstyle=\color[rgb]{0.64,0.13,0.11},
        identifierstyle=,
        commentstyle=\color[rgb]{0.27,0.27,0.28},
        stringstyle=\color[rgb]{0.82,0.16,0.14},
        morekeywords=noexcept
        }

\definecolor{hlcolor}{rgb}{0.88,0.88,0.88}
\pgfdeclareimage[interpolate=true,width=9cm]{move}{move}

\NoAutoSpaceBeforeFDP

\title{Programmation C++ Avancée}
\subtitle{Session 5 -- Patrons de Fonctions et de Classes}
\author{Joel Falcou \and Guillaume Melquiond}
\institute{Laboratoire de Recherche en Informatique}
\date{}

\subject{Computer Science}

\begin{document}
% Copyright © 2014 Lénaïc Bagnères, hnc@singularity.fr

% Licensed under the Apache License, Version 2.0 (the "License");
% you may not use this file except in compliance with the License.
% You may obtain a copy of the License at
%
% http://www.apache.org/licenses/LICENSE-2.0
%
% Unless required by applicable law or agreed to in writing, software
% distributed under the License is distributed on an "AS IS" BASIS,
% WITHOUT WARRANTIES OR CONDITIONS OF ANY KIND, either express or implied.
% See the License for the specific language governing permissions and
% limitations under the License.


% C++
\definecolor{colorkatetext}{RGB}{31,28,27}
\definecolor{colorkatetype}{RGB}{0,87,174}
\definecolor{colorkateint}{RGB}{176,128,0}
\definecolor{colorkatechar}{RGB}{146,76,157}
\definecolor{colorkatestring}{RGB}{191,3,3}
\definecolor{colorkatecomment}{RGB}{137,136,135}

\newcommand{\sourcecodecpp}{\begingroup \catcode`_=12 \sourcecodecppcmd}
\newcommand{\sourcecodecppcmd}[1]{ \lstinputlisting[style=customcpp]{#1} \endgroup}

\newcommand{\sourcecodecpplines}{\begingroup \catcode`_=12 \sourcecodecpplinescmd}
\newcommand{\sourcecodecpplinescmd}[3]{ \lstinputlisting[style=customcpp, firstline=#2, lastline=#3]{#1} \endgroup}

\newcommand{\sourcecodecpplast}{\begingroup \catcode`_=12 \sourcecodecpplastcmd}
\newcommand{\sourcecodecpplastcmd}[2]{ \lstinputlisting[style=customcpp, lastline=#2]{#1} \endgroup}

\newcommand{\sourcecodecppfirst}{\begingroup \catcode`_=12 \sourcecodecppfirstcmd}
\newcommand{\sourcecodecppfirstcmd}[2]{ \lstinputlisting[style=customcpp, firstline=#2]{#1} \endgroup}

\newcommand{\cpptext}[1]{\textcolor{colorkatetext}{\texttt{#1}}}
\newcommand{\cppkeyword}[1]{\textbf{\textcolor{colorkatetext}{\texttt{#1}}}}
\newcommand{\cpptype}[1]{\textcolor{colorkatetype}{\texttt{#1}}}
\newcommand{\cppint}[1]{\textcolor{colorkateint}{\texttt{$#1$}}}
\newcommand{\cppchar}[1]{\textcolor{colorkatechar}{\texttt{#1}}}
\newcommand{\cppstring}[1]{\textcolor{colorkatestring}{\texttt{#1}}}
\newcommand{\cppcomment}[1]{\textcolor{colorkatecomment}{\texttt{#1}}}
%%%%%%%%%%%%%%%%%%%%%%%%%%%%%%%%%%%%%%%%%%%%%%%%%%%%%%%%%%%%%%%%%%%%%%%%
%%% Template de Base
%%%%%%%%%%%%%%%%%%%%%%%%%%%%%%%%%%%%%%%%%%%%%%%%%%%%%%%%%%%%%%%%%%%%%%%%
\lstset{basicstyle=\scriptsize\ttfamily,frame=none,aboveskip=0bp,breaklines=true,belowskip=0bp,language=C++,showspaces=false, showstringspaces=false}\defverbatim[colored]\lstparamtype{%
\begin{lstlisting}
template<typename T> T maximum(T a, T b );
\end{lstlisting}}

\lstset{basicstyle=\scriptsize\ttfamily,frame=none,aboveskip=0bp,breaklines=true,belowskip=0bp,language=C++,showspaces=false, showstringspaces=false}\defverbatim[colored]\lstparaminteger{%
\begin{lstlisting}
template<int Value> struct number;
\end{lstlisting}}

\lstset{basicstyle=\scriptsize\ttfamily,frame=none,aboveskip=0bp,breaklines=true,belowskip=0bp,language=C++,showspaces=false, showstringspaces=false}\defverbatim[colored]\lstparamtmp{%
\begin{lstlisting}
template<template<class> class U, class T> 
U<T> make_container(std::size_t N, T const& value);
\end{lstlisting}}

\lstset{basicstyle=\scriptsize\ttfamily,frame=none,aboveskip=0bp,breaklines=true,belowskip=0bp,language=C++,showspaces=false, showstringspaces=false}\defverbatim[colored]\lstparamdefault{%
\begin{lstlisting}
template<class T = int, int Value = 0> struct number;
\end{lstlisting}}

\lstset{basicstyle=\scriptsize\ttfamily,frame=none,aboveskip=0bp,breaklines=true,belowskip=0bp,language=C++,showspaces=false, showstringspaces=false}\defverbatim[colored]\lsttmpfunction{%
\begin{lstlisting}
template<typename T> T maximum(T a, T b )
{
  return a > b ? a : b;
}
\end{lstlisting}}

\lstset{basicstyle=\scriptsize\ttfamily,frame=none,aboveskip=0bp,breaklines=true,belowskip=0bp,language=C++,showspaces=false, showstringspaces=false}\defverbatim[colored]\lsttmpfunctioncall{%
\begin{lstlisting}
int i,j,k;

k = maximum(i,j);
\end{lstlisting}}


\lstset{basicstyle=\scriptsize\ttfamily,frame=none,aboveskip=0bp,breaklines=true,belowskip=0bp,language=C++,showspaces=false, showstringspaces=false}\defverbatim[colored]\lstnormalswap{%
\begin{lstlisting}
void swap(int& a, int& b)
{
  int tmp(a);
  a = b;
  b = tmp;
}

void swap(float& a, float& b)
{
  float tmp(a);
  a = b;
  b = tmp;
}
\end{lstlisting}}

\lstset{basicstyle=\scriptsize\ttfamily,frame=none,aboveskip=0bp,breaklines=true,belowskip=0bp,language=C++,showspaces=false, showstringspaces=false}\defverbatim[colored]\lsttmpswap{%
\begin{lstlisting}
template<class T> void swap(T& a, T& b)
{
  int tmp(a);
  a = b;
  b = tmp;
}
\end{lstlisting}}


\lstset{basicstyle=\scriptsize\ttfamily,frame=none,aboveskip=0bp,breaklines=true,belowskip=0bp,language=C++,showspaces=false, showstringspaces=false}\defverbatim[colored]\lsttmppair{%
\begin{lstlisting}
template<class T1, class T2> struct pair
{
  pair() {}
  pair(T1 const& a, T2 const& b) : first_(a), second_(b) {}

  T1 first_;
  T2 second_;
};
\end{lstlisting}}

\lstset{basicstyle=\scriptsize\ttfamily,frame=none,aboveskip=0bp,breaklines=true,belowskip=0bp,language=C++,showspaces=false, showstringspaces=false}\defverbatim[colored]\lstemptypair{%
\begin{lstlisting}
template<class T1, class T2> struct pair
{
  pair() {}
  pair(T1 const& a, T2 const& b) : first_(a), second_(b) {}

  T1 first_;
  T2 second_;
};

template<> struct pair<void,void>
{};
\end{lstlisting}}

\lstset{basicstyle=\scriptsize\ttfamily,frame=none,aboveskip=0bp,breaklines=true,belowskip=0bp,language=C++,showspaces=false, showstringspaces=false}\defverbatim[colored]\lsttmpreturn{%
\begin{lstlisting}
template<typename T> T Fonction() { return T(); } 

int x = Fonction();      // KO
int x = Fonction<int>(); // OK

template <typename T> void Fonction2( T x1, T x2 ) {} 

int x1 = 5;
double x2 = 6.5;

Fonction( x1, x2 );                        // KO
Fonction<double>( x1, x2 );                // OK
Fonction( static_cast<double>( x1 ), x2 ); // OK
\end{lstlisting}}

\lstset{basicstyle=\scriptsize\ttfamily,frame=none,aboveskip=0bp,breaklines=true,belowskip=0bp,language=C++,showspaces=false, showstringspaces=false}\defverbatim[colored]\lstpartialspec{%
\begin{lstlisting}
template<class T> struct add_ref           
{ typedef T& type; };

template<class T> struct add_ref<T&>       
{ typedef T& type; };

template<class T> struct add_ref<T const>  
{ typedef T const& type; };

template<class T> struct add_ref<T const&> 
{ typedef T const& type; };
\end{lstlisting}}


\lstset{basicstyle=\scriptsize\ttfamily,frame=none,aboveskip=0bp,breaklines=true,belowskip=0bp,language=C++,showspaces=false, showstringspaces=false}\defverbatim[colored]\lststaticif{%
\begin{lstlisting}
template<bool Condition, typename T, typename F> class if_;

template<typename T, typename F> struct if_<true, T, F>
{
  typedef T_ type;
};

template<typename T, typename F> struct if_<false, T, F>
{
  typedef F type;
};
\end{lstlisting}}

\lstset{basicstyle=\scriptsize\ttfamily,frame=none,aboveskip=0bp,breaklines=true,belowskip=0bp,language=C++,showspaces=false, showstringspaces=false}\defverbatim[colored]\lststaticifusage{%
\begin{lstlisting}
int main()
{
  typename if_<true, int, void*>::result number(3);
  typename if_<false, int, void*>::result pointer(&number);

   typedef typename if_<(sizeof(void *) > sizeof(uint32_t))
                       , uint64_t
                       , uint32_t
                       >::type  integral_ptr_t;
	  
   integral_ptr_t ptr = reinterpret_cast<integral_ptr_t>(pointer);
}
\end{lstlisting}}

\lstset{basicstyle=\scriptsize\ttfamily,frame=none,aboveskip=0bp,breaklines=true,belowskip=0bp,language=C++,showspaces=false, showstringspaces=false}\defverbatim[colored]\lstdimension{%
\begin{lstlisting}
int main()
{
  mass<float> m = 3.4;
  distance<float> l = 4;
  duration<float> s = 0.1;

  force<float> f = m*l/(t*t);
}
\end{lstlisting}}

\lstset{basicstyle=\scriptsize\ttfamily,frame=none,aboveskip=0bp,breaklines=true,belowskip=0bp,language=C++,showspaces=false, showstringspaces=false}\defverbatim[colored]\lstisreferencea{%
\begin{lstlisting}
template<class T> struct is_reference
{
  static const bool value = false;
};
\end{lstlisting}}
\lstset{basicstyle=\scriptsize\ttfamily,frame=none,aboveskip=0bp,breaklines=true,belowskip=0bp,language=C++,showspaces=false, showstringspaces=false}\defverbatim[colored]\lstisreferenceb{%
\begin{lstlisting}
template<class T> struct is_reference<T&>
{
  static const bool value = true;
};
\end{lstlisting}}

\lstset{basicstyle=\scriptsize\ttfamily,frame=none,aboveskip=0bp,breaklines=true,belowskip=0bp,language=C++,showspaces=false, showstringspaces=false}\defverbatim[colored]\lstaddreferencea{%
\begin{lstlisting}
template<class T> struct add_reference
{
  typedef T& type;
};

template<class T> struct remove_reference
{
  typedef T type;
};
\end{lstlisting}}

\lstset{basicstyle=\scriptsize\ttfamily,frame=none,aboveskip=0bp,breaklines=true,belowskip=0bp,language=C++,showspaces=false, showstringspaces=false}\defverbatim[colored]\lstaddreferenceb{%
\begin{lstlisting}
template<class T> struct add_reference<T&>
{
  typedef T& type;
};

template<class T> struct remove_reference<T&>
{
  typedef T type;
};
\end{lstlisting}}

\lstset{basicstyle=\scriptsize\ttfamily,frame=none,aboveskip=0bp,breaklines=true,belowskip=0bp,language=C++,showspaces=false, showstringspaces=false}\defverbatim[colored]\lstsfinaea{%
\begin{lstlisting}
template< class X
        , std::size_t (X::*)() const
        > struct member {};

template<class X> static char test(member<X,&X::size>*);
\end{lstlisting}}

\lstset{basicstyle=\scriptsize\ttfamily,frame=none,aboveskip=0bp,breaklines=true,belowskip=0bp,language=C++,showspaces=false, showstringspaces=false}\defverbatim[colored]\lstsfinaeb{%
\begin{lstlisting}
template<class X> static short test(...);
\end{lstlisting}}

\lstset{basicstyle=\scriptsize\ttfamily,frame=none,aboveskip=0bp,breaklines=true,belowskip=0bp,language=C++,showspaces=false, showstringspaces=false}\defverbatim[colored]\lstsfinae{%
\begin{lstlisting}
template<typename T> struct has_size
{
  template< class X
          , std::size_t (X::*)() const
          > struct member {};

  template<class X> static char  test(member<X,&X::size>*);
  template<class X> static short test( ... );

  static const bool value  = (sizeof(test<T>(0)) == 1);
};
\end{lstlisting}}

\lstset{basicstyle=\scriptsize\ttfamily,frame=none,aboveskip=0bp,breaklines=true,belowskip=0bp,language=C++,showspaces=false, showstringspaces=false}\defverbatim[colored]\lstfacta{%
\begin{lstlisting}
template<int N> struct factorielle
{
  static const int value = N * factorielle<N-1>::value;
};
\end{lstlisting}}

\lstset{basicstyle=\scriptsize\ttfamily,frame=none,aboveskip=0bp,breaklines=true,belowskip=0bp,language=C++,showspaces=false, showstringspaces=false}\defverbatim[colored]\lstfactb{%
\begin{lstlisting}
template<> struct factorielle<0>
{
  static const int value = 1;
};
\end{lstlisting}}

\lstset{basicstyle=\scriptsize\ttfamily,frame=none,aboveskip=0bp,breaklines=true,belowskip=0bp,language=C++,showspaces=false, showstringspaces=false}\defverbatim[colored]\lstunroll{%
\begin{lstlisting}
template<int N> void remplir( int* tab )
{
  tab[N-1] = 2*(N-1)+1;
  remplir<N-1>(tab);
}

template<> void remplir<0> ( int* tab ) {}

int t[10];
remplir<10>(&t[0]);
\end{lstlisting}}


\lstset{basicstyle=\scriptsize\ttfamily,frame=none,aboveskip=0bp,breaklines=true,belowskip=0bp,language=C++,showspaces=false, showstringspaces=false}\defverbatim[colored]\lstcopy{%
\begin{lstlisting}
template<typename I1, typename I2>
void copy(I1 f, I1 l, I2 o, bool_<false> const&)
{
  while(f != l) *o++ = *f++;
}

template<typename T>
void copy(const T* f, const T* l, T* o, bool_<true> const&)
{
  memmove(o, f, (l-f)*sizeof(T));
}

template<typename I1, typename I2>
void copy(I1 f, I1 l, I2 o)
{
 typedef typename std::iterator_traits<I1>::value_type value_type;
 copy(f, l, o, has_trivial_assign<value_type>());
}
\end{lstlisting}}


\begin{frame}[plain]
\titlepage
\end{frame}
\setcounter{framenumber}{0}
\frame
{
  \frametitle{Contexte}
  \begin{block}{Qu'est-ce qu'un \texttt{template} ?}
  \begin{itemize}
  \footnotesize
  \item Modèle générique de code de fonction ou de classe
  \item Paramétrable par des types abstraite ou des valeurs constantes
  \item Spécifier ces paramètres \textbf{instantie} le \texttt{template} et génére du code
  \end{itemize}
  \end{block}

  \begin{block}{Pourquoi les templates en C++ ?}
  \begin{itemize}
  \item Base de la généricité en C++
  \item Support pour le polymorphisme statique
  \item Aller au delà des macros du préprocesseurs
  \end{itemize}
  \end{block}
}
  
\frame
{
  \frametitle{Principe de bases}
  \begin{block}{Syntaxe}
  \begin{itemize}
  \item Paramètrage introduit par : \texttt{template<>}
  \item Chaque paramètres peut être:
  \begin{itemize}
  \item \textbf{un type} introduit par le mot clé \texttt{class} ou \texttt{typename}
  \lstparamtype
  \item \textbf{une valeur} de type entière
  \lstparaminteger
  \item \textbf{un type template} déclaré \textit{in situ}
  \lstparamtmp  
  \item Ces paramètres peuvent avoir un valeur par défaut
  \lstparamdefault
  \end{itemize}
  \end{itemize}
  \end{block}
}

\frame
{
  \frametitle{Fonction template}
  \begin{block}{Principes}
  \begin{itemize}
  \footnotesize
  \item Une \textbf{fonction template} est un modèle de génération de fonction paramétrable
  \item Elle remplace avantageusement les macros en étant type-safe
  \vspace{0.5cm}
  \lsttmpfunction
  \item L'appel d'une \textbf{fonction template} s'effectue comme pour une fonction normale
  \item Le compilateur \textbf{infére} des arguments le type des paramètres.
  \vspace{0.5cm}
  \lsttmpfunctioncall
  \end{itemize}
  \end{block}
}

\frame
{
  \frametitle{Fonction template}
  \begin{block}{Exemple : la fonction \texttt{swap}}
  \only<1>{ \lstnormalswap }
  \only<2>{ \lsttmpswap    }
  \end{block}
}


\frame
{
  \frametitle{Fonction template}
  \begin{block}{Algorithme de résolution d'un appel \texttt{template}}
  \footnotesize Quand le compilateur détecte un appel de fonction il suit l'algorithme suivant pour associer la bonne fonction.
  \begin{enumerate}
  \footnotesize 
  \item Dans les fonctions non templates:
  \begin{enumerate}
  \footnotesize 
  \item Une seule association exacte : \alert{résolution terminée}
  \item Plusieurs association exacte : \alert{erreur}
  \item Aucune : \alert{Allez en 2}
  \end{enumerate}
  \item Dans les fonctions templates:
  \begin{enumerate}
  \footnotesize 
  \item Une seule association exacte : \alert{résolution terminée}
  \item Plusieurs association exacte : \alert{erreur}
  \item Aucune : \alert{Allez en 3}
  \end{enumerate}
  \item Réexaminer les non templates de façons classiques avec d'éventuelles conversions de type.
 \end{enumerate}
 \end{block}
}

\frame
{
  \frametitle{Fonction template}
  \begin{block}{Cas particulier - Retour template}
  \lsttmpreturn
  \end{block}
}
\frame
{
  \frametitle{Templates variadiques}
  \begin{block}{Applications aux fonctions}
  \begin{itemize}
  \item Remplacement type-safe du \cpptext{...} du C
  \item Notion de \textit{paramaters pack}
  \item Déduction automatique des types
  \end{itemize}
  
  \only<1>{\sourcecodecpplines{code/vararg.cpp}{3}{6}}
  \only<2>{\sourcecodecpplines{code/vararg.cpp}{8}{12}}  
  \only<3>{\sourcecodecpplines{code/vararg.cpp}{14}{21}}  
  \end{block}{}
}

\frame
{
  \frametitle{Fonctions et inférence de type}
  \begin{block}{Impact sur le retour des fonctions}
  \begin{itemize}
  \item \texttt{auto} et \texttt{decltype} simplifient l'\'ecriture
  du prototype des fonctions
  \item Notion de \textit{trailing return type}
  \end{itemize}

  \only<1>{\sourcecodecpp{code/trail03.cpp}}
  \only<2>{\sourcecodecpp{code/trail11.cpp}}
  \only<3>{\sourcecodecpp{code/trail14.cpp}}
  
  \end{block}{}
}
\frame
{
  \frametitle{Classe template}
  \begin{block}{Principes}
  \begin{itemize}
  \footnotesize
  \item Une \textbf{classe \texttt{template}} est un modèle de génération de classe paramétrable
  \item Elle permet de gérer des variantes de classes sans polymorphisme dynamique
  \item Les paramètres \texttt{template}[s] sont à spécifier explicitement.
  \item Une classe \texttt{template} ne devient un type complet que lorsqu'elle est entièrement spécifiée.
  \end{itemize}
  \end{block}

  \begin{block}{Quelques détails ...}
  \begin{itemize}
  \footnotesize
  \item Une classe non \texttt{template} peut avoir une ou plusieurs méthodes \texttt{template}[s]
  \item Une classe \texttt{template} peut avoir une ou plusieurs méthodes \texttt{template}[s] paramétrés sur un autre jeu de type abstrait
  \item Si \texttt{A} hérite de \texttt{B}, et que \texttt{C} est une classe \texttt{template}, il n'existe aucun lien implicite
  entre \texttt{C<A>} et \texttt{C<B>}.
  \end{itemize}
  \end{block}
}

\frame
{
  \frametitle{Classe template}
  \begin{block}{Exemple : la classe \texttt{pair<T1,T2>}}

  \lsttmppair

  \end{block}
}

\frame
{
  \frametitle{Atelier Pratique}
  \begin{block}{La classe \texttt{fixed\_array<T,N>}}
  \begin{itemize}
  \item \texttt{fixed\_array<T,N>} représente un tableau de \texttt{N} éléments de type T
  \item Proposez une implantation simple fournissant les opérations classiques sur les tableaux
  \end{itemize}
  \end{block}

  \begin{block}{Indices}
  \begin{itemize}
  \item \texttt{T} et \texttt{N} permettent de reconstruire un \texttt{T tab[N]}
  \item Que deviennent \texttt{size}, \texttt{empty} etc ...
  \end{itemize}
  \end{block}
}

\frame
{
  \frametitle{Spécialisation de template}
  \begin{block}{Cas d'utilisation}
  \begin{itemize}
  \item Quid d'un \texttt{fixed\_array<T,0>} ?
  \item Specifier des comportement differents en fonctions du paramètrage
  \item Version statique du patron de conception "Strategy" ou "State"
  \end{itemize}
  \end{block}

  \begin{block}{Mise en \oe{uvre}}
  \begin{itemize}
  \item Spécialisation totale
  \item Spécialisation partielle
  \end{itemize}
  \end{block}{}
}

\frame
{
  \frametitle{Spécialisation de template}
  \begin{block}{Spécialisation totale}
  Permet de spécifier un \texttt{template} entièrement pour un jeu de paramètre donné.
  \vspace{0.5cm}
  \lstemptypair
  \end{block}
}

\frame
{
  \frametitle{Spécialisation de template}
  \begin{block}{Spécialisation partielle}
  Permet de spécifier certain arguments du \texttt{template} afin de spécialiser une partie de son comportement
  \lstpartialspec
  \begin{center}\alert{Attention aux ambiguités !}\end{center}
  \end{block}
}

\frame
{
  \frametitle{Spécialisation de template}
  \begin{block}{Selecteur conditionnel de type}
  \only<1>
  {
    \begin{block}{Spécifications}
    \texttt{if\_<Bool,T1,T2>::type} s'évalue en \texttt{T1} si \texttt{Bool == true} et \texttt{T12} sinon.
    \end{block}
  }
  \only<2>{ \lststaticif }
  \only<3>{ \lststaticifusage }
  \end{block}
}

\end{document}
