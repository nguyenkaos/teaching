% !TEX encoding = UTF-8 Unicode
% -*- coding: UTF-8; -*-
% vim: set fenc=utf-8
\documentclass[svgnames]{beamer}
\setlength{\tabcolsep}{3em}
\usetheme[pageofpages=of,% String used between the current page and the
                         % total page count.
          titleline=true,
          alternativetitlepage=true% Use the fancy title page.
          ]{Torino}
\usecolortheme{metascale}
\usepackage{listings}

\ifxetex
  \usepackage{fontspec}
  \defaultfontfeatures{Mapping=tex-text}
  \setsansfont[ItalicFont={GillSansMTPro-BookItalic}]{GillSansMTPro-Book}
  \setmonofont{Inconsolata}
  \newcommand{\codefont}{\ttfamily}
\else
  \usepackage[utf8x]{inputenc}
  %\usepackage[nott]{inconsolata}
  %\newcommand{\codefont}{\fontfamily{fi4}\selectfont}
  \usepackage{inconsolata}
  \newcommand{\codefont}{\ttfamily}
\fi
\usepackage{graphicx}
\usepackage{color}
%\usepackage{multicol}
%\usepackage{array}
%\usepackage{colortbl}
\usepackage[frenchb]{babel}
%\usepackage{pgfplots}
%\usepackage{standalone}
\usepackage{relsize}

% setup tikz
%% \usepackage{tikz}
%% \usetikzlibrary{calc,trees,positioning,arrows,chains,shapes.geometric,%
%% decorations.pathreplacing,decorations.pathmorphing,shapes,%
%% matrix,shapes.symbols,plotmarks,decorations.markings,shadows,%
%% snakes,backgrounds}
\def\print#1{\pgfmathparse{#1}\pgfmathresult}
\def\colx{Blue!40}
\def\coly{Blue!20}
\def\colz{white}

\def\C++{\textup{C}\nolinebreak[4]\hspace{-.05em}\raisebox{.4ex}{\relsize{-3}{\textbf{++}}}}

% vertical align box
\newcommand*{\vcenteredhbox}[1]{\begingroup
\setbox0=\hbox{#1}\parbox{\wd0}{\box0}\endgroup}

\lstdefinestyle{customcpp}{language=C++,
        basicstyle=\footnotesize\codefont,
        tabsize=2,
        numberstyle=\footnotesize,
        showstringspaces=false,
        %columns=fullflexible,
        keywordstyle=\color[rgb]{0.64,0.13,0.11},
        identifierstyle=,
        commentstyle=\color[rgb]{0.27,0.27,0.28},
        stringstyle=\color[rgb]{0.82,0.16,0.14},
        morekeywords=noexcept
        }

\definecolor{hlcolor}{rgb}{0.88,0.88,0.88}
\pgfdeclareimage[interpolate=true,width=9cm]{move}{move}

\NoAutoSpaceBeforeFDP

\title{Programmation C++ Avancée}
\subtitle{Session 4 -- Gestions des Erreurs}
\author{Joel Falcou \and Guillaume Melquiond}
\institute{Laboratoire de Recherche en Informatique}
\date{}

\subject{Computer Science}

\begin{document}
% Copyright © 2014 Lénaïc Bagnères, hnc@singularity.fr

% Licensed under the Apache License, Version 2.0 (the "License");
% you may not use this file except in compliance with the License.
% You may obtain a copy of the License at
%
% http://www.apache.org/licenses/LICENSE-2.0
%
% Unless required by applicable law or agreed to in writing, software
% distributed under the License is distributed on an "AS IS" BASIS,
% WITHOUT WARRANTIES OR CONDITIONS OF ANY KIND, either express or implied.
% See the License for the specific language governing permissions and
% limitations under the License.


% C++
\definecolor{colorkatetext}{RGB}{31,28,27}
\definecolor{colorkatetype}{RGB}{0,87,174}
\definecolor{colorkateint}{RGB}{176,128,0}
\definecolor{colorkatechar}{RGB}{146,76,157}
\definecolor{colorkatestring}{RGB}{191,3,3}
\definecolor{colorkatecomment}{RGB}{137,136,135}

\newcommand{\sourcecodecpp}{\begingroup \catcode`_=12 \sourcecodecppcmd}
\newcommand{\sourcecodecppcmd}[1]{ \lstinputlisting[style=customcpp]{#1} \endgroup}

\newcommand{\sourcecodecpplines}{\begingroup \catcode`_=12 \sourcecodecpplinescmd}
\newcommand{\sourcecodecpplinescmd}[3]{ \lstinputlisting[style=customcpp, firstline=#2, lastline=#3]{#1} \endgroup}

\newcommand{\sourcecodecpplast}{\begingroup \catcode`_=12 \sourcecodecpplastcmd}
\newcommand{\sourcecodecpplastcmd}[2]{ \lstinputlisting[style=customcpp, lastline=#2]{#1} \endgroup}

\newcommand{\sourcecodecppfirst}{\begingroup \catcode`_=12 \sourcecodecppfirstcmd}
\newcommand{\sourcecodecppfirstcmd}[2]{ \lstinputlisting[style=customcpp, firstline=#2]{#1} \endgroup}

\newcommand{\cpptext}[1]{\textcolor{colorkatetext}{\texttt{#1}}}
\newcommand{\cppkeyword}[1]{\textbf{\textcolor{colorkatetext}{\texttt{#1}}}}
\newcommand{\cpptype}[1]{\textcolor{colorkatetype}{\texttt{#1}}}
\newcommand{\cppint}[1]{\textcolor{colorkateint}{\texttt{$#1$}}}
\newcommand{\cppchar}[1]{\textcolor{colorkatechar}{\texttt{#1}}}
\newcommand{\cppstring}[1]{\textcolor{colorkatestring}{\texttt{#1}}}
\newcommand{\cppcomment}[1]{\textcolor{colorkatecomment}{\texttt{#1}}}

\begin{frame}[plain]
\titlepage
\end{frame}
\setcounter{framenumber}{0}

\frame
{
  \frametitle{Gestion des Erreurs}
  \begin{block}{Quoi et Pourquoi}
  \begin{itemize}
  \item Valider l'état d'un objet avant l'appel d'une fonction
  \item Rapporter des erreurs inattendues
  \item Rapporter des événements indépendant du code
  \end{itemize}
  \end{block}
 
  \begin{block}{Mise en \oe{uvre}}
  \begin{itemize}
  \item Assertion
  \item Exception
  \end{itemize}
  \end{block}
}

\frame
{
  \frametitle{Assertion}
  \begin{block}{Principe}
  \begin{itemize}
  \item Permet de valider les \alert{pré-conditions} d'une fonction
  \item Fournie par \cpptext{<cassert>}
  \item Termine le programme en Debug
  \item Disparaît en Release
  \end{itemize}
  \end{block}

  \begin{overlayarea}{\textwidth}{4cm}
  \only<1>{\sourcecodecpplines{code/assert.cpp}{1}{7}}
  \only<2>{\sourcecodecpplines{code/assert.cpp}{9}{13}}
  \end{overlayarea}
}

\frame
{
  \frametitle{Exception}
  \begin{block}{Exceptions définies par les utilisateurs}
  \begin{itemize}
  %\item Valide les \alert{post-conditions} d'une fonction
  \item Effectue une sortie anticipée de portée
  \item Une exception = un type
  \end{itemize}
  \end{block}

  \begin{overlayarea}{\textwidth}{5cm}
  \only<1>{\sourcecodecpplines{code/trycatch.cpp}{1}{6}}
  \only<2>{\sourcecodecpplines{code/trycatch.cpp}{8}{18}}
  \end{overlayarea}
}

\frame
{
  \frametitle{Garantie d'Exception}
  \begin{block}{Principe}
  \begin{itemize}
  \item Valide les \alert{post-conditions exceptionnelles} d'une fonction
  \item Termine le programme en Debug
  \item Disparaît en Release
  \end{itemize}
  \end{block}

  \begin{block}{Quelles garanties ?}
  \begin{itemize}
  \item NOEXCEPT
  \item STRONG
  \item BASIC
  \end{itemize}
  \end{block}
}


\frame
{
  \frametitle{Garantie d'Exception}
  \begin{block}{Niveau NOEXCEPT}
  \begin{itemize}
  \item Comportement normal garanti dans tous les cas
  \item Aucune exception ne sera émise
  \item Les exceptions internes ne sont pas observables
  \item Mot-clé \cppkeyword{noexcept}
  \end{itemize}
  \end{block}

  \begin{overlayarea}{\textwidth}{3cm}
  \only<1>{\sourcecodecpplines{code/noexcept.cpp}{1}{6}}
  \only<2>{\sourcecodecpplines{code/noexcept.cpp}{8}{13}}
  \only<3>{\sourcecodecpplines{code/noexcept.cpp}{15}{18}}
  \only<4>{\sourcecodecpplines{code/noexcept.cpp}{20}{23}}
  \end{overlayarea}
}

\frame
{
  \frametitle{Garantie d'Exception}
  \begin{block}{Niveau STRONG}
  \begin{itemize}
  \item Comportement exceptionnel permis
  \item En cas d'exception, les données originelles ne sont pas modifiées
  \item L'objet reste valide et cohérent
  \end{itemize}
  \end{block}
}

\frame
{
  \frametitle{Garantie d'Exception}
  \begin{block}{Niveau BASIC}
  \begin{itemize}
  \item Une exécution partielle de la fonction est possible
  \item Des effets de bords peuvent avoir lieu
  \item Aucune fuite de ressources
  \item L'état de l'objet est valide mais pas forcément cohérent
  \end{itemize}
  \end{block}
}

\frame
{
  \frametitle{Mise en pratique}
  \begin{overlayarea}{\textwidth}{5cm}
  \only<1>{\sourcecodecpplines{code/assign.cpp}{1}{10}}
  \only<2>{\sourcecodecpplines{code/assign.cpp}{12}{21}}
  \only<3>{\sourcecodecpplines{code/copy.cpp}{1}{7}}
  \only<4>{\sourcecodecpplines{code/copy.cpp}{9}{16}}
  \end{overlayarea}
}

\frame
{
  \frametitle{RAII et capture}
  \sourcecodecpp{code/nocapture.cpp}
}


\end{document}
