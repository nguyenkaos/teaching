% !TEX encoding = UTF-8 Unicode
% -*- coding: UTF-8; -*-
% vim: set fenc=utf-8
\documentclass[svgnames]{beamer}
\setlength{\tabcolsep}{3em}
\usetheme[pageofpages=of,% String used between the current page and the
                         % total page count.
          titleline=true,
          alternativetitlepage=true% Use the fancy title page.
          ]{Torino}
\usecolortheme{metascale}
\usepackage{listings}

\ifxetex
  \usepackage{fontspec}
  \defaultfontfeatures{Mapping=tex-text}
  \setsansfont[ItalicFont={GillSansMTPro-BookItalic}]{GillSansMTPro-Book}
  \setmonofont{Inconsolata}
  \newcommand{\codefont}{\ttfamily}
\else
  \usepackage[utf8x]{inputenc}
  \usepackage[nott]{inconsolata}
  \newcommand{\codefont}{\fontfamily{fi4}\selectfont}
\fi
\usepackage{graphicx}
\usepackage{color}
\usepackage{multicol}
\usepackage{array}
\usepackage{colortbl}
\usepackage[frenchb]{babel}
\usepackage{pgfplots}
\usepackage{standalone}

% setup tikz
\usepackage{tikz}
\usetikzlibrary{calc,trees,positioning,arrows,chains,shapes.geometric,%
decorations.pathreplacing,decorations.pathmorphing,shapes,%
matrix,shapes.symbols,plotmarks,decorations.markings,shadows,%
snakes,backgrounds}
\def\print#1{\pgfmathparse{#1}\pgfmathresult}
\def\colx{Blue!40}
\def\coly{Blue!20}
\def\colz{white}

\providecommand{\cpp}[1][~]{\textsc{C++}#1}

% vertical align box
\newcommand*{\vcenteredhbox}[1]{\begingroup
\setbox0=\hbox{#1}\parbox{\wd0}{\box0}\endgroup}

\lstdefinestyle{customcpp}{language=C++,
        basicstyle=\footnotesize\codefont,
        tabsize=2,
        numberstyle=\footnotesize,
        showstringspaces=false,
        columns=fullflexible,
        keywordstyle=\color[rgb]{0.64,0.13,0.11},
        identifierstyle=,
        commentstyle=\color[rgb]{0.27,0.27,0.28},
        stringstyle=\color[rgb]{0.82,0.16,0.14},
        }

\definecolor{hlcolor}{rgb}{0.88,0.88,0.88}

\title{Programmer en C++}
\subtitle{Session 1 - Types et Fonctions}
\author{Joel Falcou \and Guillaume Melquiond}
\institute{Laboratoire de Recherche en Informatique}
\date{}

\subject{Computer Science}

\begin{document}
% Copyright © 2014 Lénaïc Bagnères, hnc@singularity.fr

% Licensed under the Apache License, Version 2.0 (the "License");
% you may not use this file except in compliance with the License.
% You may obtain a copy of the License at
%
% http://www.apache.org/licenses/LICENSE-2.0
%
% Unless required by applicable law or agreed to in writing, software
% distributed under the License is distributed on an "AS IS" BASIS,
% WITHOUT WARRANTIES OR CONDITIONS OF ANY KIND, either express or implied.
% See the License for the specific language governing permissions and
% limitations under the License.


% C++
\definecolor{colorkatetext}{RGB}{31,28,27}
\definecolor{colorkatetype}{RGB}{0,87,174}
\definecolor{colorkateint}{RGB}{176,128,0}
\definecolor{colorkatechar}{RGB}{146,76,157}
\definecolor{colorkatestring}{RGB}{191,3,3}
\definecolor{colorkatecomment}{RGB}{137,136,135}

\newcommand{\sourcecodecpp}{\begingroup \catcode`_=12 \sourcecodecppcmd}
\newcommand{\sourcecodecppcmd}[1]{ \lstinputlisting[style=customcpp]{#1} \endgroup}

\newcommand{\sourcecodecpplines}{\begingroup \catcode`_=12 \sourcecodecpplinescmd}
\newcommand{\sourcecodecpplinescmd}[3]{ \lstinputlisting[style=customcpp, firstline=#2, lastline=#3]{#1} \endgroup}

\newcommand{\sourcecodecpplast}{\begingroup \catcode`_=12 \sourcecodecpplastcmd}
\newcommand{\sourcecodecpplastcmd}[2]{ \lstinputlisting[style=customcpp, lastline=#2]{#1} \endgroup}

\newcommand{\sourcecodecppfirst}{\begingroup \catcode`_=12 \sourcecodecppfirstcmd}
\newcommand{\sourcecodecppfirstcmd}[2]{ \lstinputlisting[style=customcpp, firstline=#2]{#1} \endgroup}

\newcommand{\cpptext}[1]{\textcolor{colorkatetext}{\texttt{#1}}}
\newcommand{\cppkeyword}[1]{\textbf{\textcolor{colorkatetext}{\texttt{#1}}}}
\newcommand{\cpptype}[1]{\textcolor{colorkatetype}{\texttt{#1}}}
\newcommand{\cppint}[1]{\textcolor{colorkateint}{\texttt{$#1$}}}
\newcommand{\cppchar}[1]{\textcolor{colorkatechar}{\texttt{#1}}}
\newcommand{\cppstring}[1]{\textcolor{colorkatestring}{\texttt{#1}}}
\newcommand{\cppcomment}[1]{\textcolor{colorkatecomment}{\texttt{#1}}}

\begin{frame}[plain]
\titlepage
\end{frame}
\setcounter{framenumber}{0}

%%%%%%%%%%%%%%%%%% INTRO %%%%%%%%%%%%%%%%%%%%
\section{Introduction}
\frame
{
  \frametitle{Un peu d'histoire ...}
  \begin{block}{C++ \`a travers les \^ages}
  \begin{itemize}
  \footnotesize
  \item 1978 Bjarne Stroustrup travaille sur Simula67 qui s'av\'ere peu efficace
  \item 1980 D\'emarrage du projet \textit{C with Classes}, du C orient\'e objet compil\'e via \texttt{CFront}.
  \item 1983 \textit{C with Classes} devient \texttt{C++}
  \item 1985 Parution de \textit{The C++ Programming Language}
  \item 1998 Premi\`ere standarisation de \texttt{C++} : ISO/IEC 14882:1998
  \item 2005 Parution du \textit{C++ Technical Report 1} qui deviendra C++0x
  \item 2011 Ratification de C++0x sous le nom C++11
  \item 2014 Ratification de C++14
  \item 2017,2020,... Prochaines \textit{milestones} du langage
  \end{itemize}
  \end{block}
}

\frame
{
  \frametitle{C++ et C -- Une histoire de famille}
  \begin{block}{C++ comme h\'eritier du C}
  \begin{itemize}
  \item C++ est un sur-ensemble de C
  \item C++ tente de minimiser les discordances
  \item Possibilit\'e de conserver une compatibilit\'e binaire
  \end{itemize}
  \end{block}

  \begin{block}{C++, ce fils rebel}
  \begin{itemize}
  \item C++ \'evolue afin de d\'elaisser les \'el\'ements fondamentaux de C
  \item Changement drastique du mod\'ele de programmation
  \item Vers un langage quasi-fonctionnel plus qu'objet et imp\'eratif
  \end{itemize}
  \end{block}
}

%%%%%%%%%%%%%%%%%% TYPES %%%%%%%%%%%%%%%%%%%%
\section{Rappels}
\begin{frame}[plain]
\begin{center}\Huge Rappels\end{center}
\end{frame}

\frame
{
  \frametitle{Types fondamentaux}
  \begin{block}{Types numériques}
  \begin{itemize}
  \footnotesize
  \item Entiers : \cpptype{(unsigned) char},\cpptype{(unsigned) short},\cpptype{(unsigned) int}
  \item Entiers portables : \cpptype{std::ptrdiff\_t}, \cpptype{std::uintptr\_t}, \cpptype{std::size\_t}
  \item Réels IEEE754 : \cpptype{float}, \cpptype{double}
  \item Booléen : \cpptype{bool} de valeur \cppint{true} ou \cppint{false}
  \end{itemize}

  \only<1>{\sourcecodecpplines{code/basic/types.cpp}{6}{12}}
  \only<2>{\sourcecodecpplines{code/basic/types.cpp}{22}{29}}
  \only<3>{\sourcecodecpplines{code/basic/types.cpp}{40}{44}}
  \only<4>{\sourcecodecpplines{code/basic/types.cpp}{55}{60}}    
  \end{block}{}
}

\frame
{
  \frametitle{Types fondamentaux}
  \begin{block}{Qualificateurs de Types}
  \begin{itemize}
  \item \cppkeyword{const} : exprime la constance d'une valeur
  \item \cppkeyword{*} : indique un type \textbf{pointeur}
  \item \cppkeyword{\&} : indique un type \textbf{référence}
  \end{itemize}

  \only<1>{\sourcecodecpplines{code/basic/qualificator.cpp}{5}{9}}
  \only<2>{\sourcecodecpplines{code/basic/qualificator.cpp}{11}{15}}
  \only<3>{\sourcecodecpp{code/basic/pointer_question.cpp}}
  \only<4>{\sourcecodecpplines{code/basic/qualificator.cpp}{17}{21}}
  \end{block}{}
}

\frame
{
  \frametitle{Types et variables}
  \begin{block}{Chaîne de caractères}
  \begin{itemize}
  \item Représentée par le type standard \cpptype{std::string}
  \item Accessible via \cpptext{\#include <string>}
  \item Comparable, copiable, concatenable, redimensionnable
  \item Compatible avec les chaînes C
  \end{itemize}  

  \only<1>{\sourcecodecpplines{code/basic/string.cpp}{6}{12}}
  \only<2>{\sourcecodecpplines{code/basic/string.cpp}{19}{22}}
  \only<3>{\sourcecodecpplines{code/basic/string.cpp}{26}{27}}
  \end{block}{}
}

\frame
{
  \frametitle{Types et variables}
  \begin{block}{Tableau dynamique}
  \begin{itemize}
  \item Représenté par le type standard \cpptype{std::vector<T>}
  \item Accessible via \cpptext{\#include <vector>}
  \item Comparable, copiable, redimensionnable
  \item Optimisé pour gérer tout type de contenu
  \end{itemize}    

  \sourcecodecpplines{code/basic/vector.cpp}{6}{13}
  \end{block}{}
}

\frame
{
  \frametitle{Types et variables}
  \begin{block}{Tableau statique}
  \begin{itemize}
  \item Représenté par le type standard \cpptype{std::array<T,N>}
  \item Accessible via \cpptext{\#include <array>}
  \item Propose une sémantique de premier ordre
  \item Binairement équivalent à \cpptext{T[N]}
  \end{itemize}    

  \only<1>{\sourcecodecpplines{code/basic/stdarray.cpp}{15}{21}}
  \only<2>{\sourcecodecpplines{code/basic/stdarray.cpp}{4}{11}}  
  \end{block}{}
}

\frame
{
  \frametitle{Types de données abstraits}
  \begin{block}{Énumération}
  \begin{itemize}
  \footnotesize
  \item Encapsulation d'un ensemble discret de valeurs partageant une sémantique
  \item C++ permet le typage des \cppkeyword{enum}
  \item C++ permet la spécification du support de type
  \end{itemize}
  
  \only<1>{\sourcecodecpplines{code/basic/enum.cpp}{1}{7}}
  \only<2>{\sourcecodecpplines{code/basic/enum.cpp}{9}{12}}
  \only<3>{\sourcecodecpplines{code/basic/enum.cpp}{14}{15}}
  \only<4>{\sourcecodecpplines{code/basic/enum.cpp}{17}{21}}
  \only<5>{\sourcecodecpplines{code/basic/enum.cpp}{23}{24}}
  \end{block}{}
}

\frame
{
  \frametitle{Types de données abstraits}
  \begin{block}{Paire et Tuple}
  \begin{itemize}
  \item \cpptype{std::pair} : deux membres de types quelconques
  \item \cpptype{std::tuple} : généralisation de \cpptype{pair}
  \item Copiable, assignable, intropectable
  \end{itemize}
    
  \only<1>{\sourcecodecpplines{code/basic/tuple.cpp}{7}{9}}
  \only<2>{\sourcecodecpplines{code/basic/tuple.cpp}{11}{20}}
  \only<3>{\sourcecodecpplines{code/basic/tuple.cpp}{21}{27}}

  \end{block}{}
}

%%%%%%%%%%%%%%%%%% FONCTIONS %%%%%%%%%%%%%%%%%%%%
\section{Aspect Imp\'eratif}
\begin{frame}[plain]
\begin{center}\Huge Aspect Imp\'eratif\end{center}
\end{frame}

\frame
{
  \frametitle{Fonctions et surcharge}
  \begin{block}{Notion de fonction}
  \begin{itemize}
  \item Groupe nommé de \cpptext{statements} appelable à volonté
  \item Élément fondamental de l'encapsulation en C et en C++
  \item Notion de paramètres et de valeur de retour
  \end{itemize}
  \end{block}{}

\only<1>
{
  \begin{block}{Déclaration d'une fonction}\footnotesize
  \begin{center}\footnotesize\cpptext{type name( parameter1, parameter2, ...) \{ statements \}}\end{center}
  \vspace{-0.5cm}
  \begin{itemize}\footnotesize
  \item \cpptext{type} : Type de la valeur retournée par la fonction
  \item \cpptext{name} : Identifiant de la fonction
  \item \cpptext{parameter*} : Transfert d'information du point d'appel à la fonction
  \item \cpptext{statements} : Corps de la fonction, \textit{i.e} le code effectif de la fonction
  \end{itemize}
  \end{block}{}
}

  \only<2>{\sourcecodecpplines{code/basic/functions.cpp}{3}{6}}
  \only<3>{\sourcecodecpplines{code/basic/functions.cpp}{8}{11}}
  \only<4>{\sourcecodecpplines{code/basic/functions.cpp}{13}{16}}
  \only<5>{\sourcecodecpplines{code/basic/functions.cpp}{18}{24}}
  \only<6>{\sourcecodecpplines{code/basic/functions.cpp}{26}{31}}    
}

\frame
{
  \frametitle{Fonctions et surcharge}
  \begin{block}{Définition}
  \begin{itemize}\footnotesize
  \item Forme de polymorphisme ad-hoc
  \item En C : une fonction = un symbole
  \item En \cpp : une fonction = une signature
  \item Une signature = symbole + type des paramètres + qualificateur
  \end{itemize}
  \end{block}

  \begin{block}{Exemples:}
  \begin{itemize}
  \item \cpptext{double f()}
  \item \cpptext{double f(int)} - OK
  \item \cpptext{double f(double,int)} - OK
  \item \cpptext{double f(...)} - OK
  \item \cpptext{int f()} - KO
  \end{itemize}
  \end{block}
}

\frame
{
  \frametitle{Fonctions et surcharge}
  \begin{block}{Fonctions génériques}
  \begin{itemize}
  \item Généralisation de la surcharge de fonction
  \item Généralisation/abstraction des types des paramètres
  \item Déduction automatique des types
  \end{itemize}
  
  \only<1>{\sourcecodecpplines{code/basic/overload.cpp}{18}{21}}
  \only<2>{\sourcecodecpplines{code/basic/overload.cpp}{25}{28}}  
  \end{block}{}
}

\frame
{
  \frametitle{Surcharge des opérateurs}
  \begin{block}{Objectifs}
  \begin{itemize}
  \item Rendre une classe similaire à un type de base
  \item Renforcer la sémantique et simplifier la syntaxe
  \item Attention aux abus !
  \end{itemize}
  \end{block}{}

  \begin{block}{Syntaxe}
  \begin{itemize}
  \item Operateur unaire membre : \cpptext{type type::operator!()}
  \item Operateur binaire membre : \cpptext{type type::operator+(type)}
  \item Operateur unaire : \cpptext{type operator-(type)}
  \item Operateur binaire : \cpptext{type operator+(type,type)}
  \end{itemize}
  \end{block}{}
}

\frame
{
  \frametitle{Surcharge des opérateurs}
  \only<1>{\sourcecodecpplines{code/poo/operators.cpp}{1}{16}}
  \only<2>{\sourcecodecpplines{code/poo/operators.cpp}{18}{22}}  
}


\frame
{
  \frametitle{Fonctions et surcharge}
  \begin{block}{Fonctions génériques}
  \begin{itemize}
  \item Généralisation de la surcharge de fonction
  \item Généralisation/abstraction des types des paramètres
  \item Déduction automatique des types
  \end{itemize}
  
  \only<1>{\sourcecodecpplines{code/basic/overload.cpp}{18}{21}}
  \only<2>{\sourcecodecpplines{code/basic/overload.cpp}{25}{28}}  
  \end{block}{}
}

\begin{frame}
\frametitle{Règles de résolution}
\begin{block}{Processus Général [1]}
\footnotesize
\begin{itemize}
\item Les variantes du symbole sont recherchées pour créer l'\emph{Overload Set} ($\Omega$).
\item Toute variante n'ayant pas le nombre de paramètres adéquat est éliminée pour obtenir le \emph{Viable Set}.
\item On recherche dans cet ensemble la \emph{Best Viable Function}.
\item On vérifie l'accessibilité et l'unicité de la sélection.
\end{itemize}
\end{block}

\begin{block}{Que faire de tout ça ?}
\footnotesize
\begin{itemize}
\item Comment définir ($\Omega$) ?
\item Quels critères pour la \emph{Best Viable Function} ?
\end{itemize}
\end{block}

\tiny{[1] \textit{C++ Templates: The Complete Guide} -- David Vandevoorde, Nicolai M. Josuttis}

\end{frame}

\begin{frame}
\frametitle{Règles de résolution}
\begin{block}{Construction de $\Omega$}
\begin{itemize}\footnotesize
\item Ajouter toutes les fonctions non-template avec le bon nom
\item Ajouter les variantes templates une fois la substitution des paramètres templates est 
effectuée avec succés
\item $\Omega$ est un treillis: les fonctions non-templates dominent les fonctions template
\end{itemize}
\end{block}

\begin{block}{Sélection de la \emph{Best Viable Function}}
\begin{itemize}
\footnotesize
\item Chaque argument est associé à un Séquence de Conversion Implicite (ICS)
\item Chaque argument est trié par rapport à son ICS
\item Si un argument n'est pas triable, le compilateur boude.
\end{itemize}
\end{block}

\end{frame}

\begin{frame}
\frametitle{Règles de résolution}
\begin{block}{Les Séquences de Conversion Implicite}
\begin{itemize}\footnotesize
\item Conversions standards (SCS)
\begin{itemize}
\footnotesize
\item Correspondance exacte
\item Promotion
\item Conversion
\end{itemize}
\item Conversion utilisateur (UDCS) , composée comme:
\begin{itemize}
\footnotesize
\item une séquence standard
\item une conversion définie par l'utilisateur
\item une deuxiéme séquence standard
\item Un UDCS est meilleur qu'un autre si sa seconde SCS est meilleure que l'autre
\end{itemize}

\item Séquence de conversion par ellipse C
\end{itemize}
\end{block}
\end{frame}

\begin{frame}
\frametitle{Règles de résolution}
\only<1>{\sourcecodecpp{code/imperatif/overload1.cpp}}
\only<2>{\sourcecodecpp{code/imperatif/overload2.cpp}}

\begin{block}{Résultats}
\only<1>
{
\begin{itemize}
\footnotesize
\item \cpptext{f(1)} $\rightarrow$ \cpptext{void f(int)}
\item \cpptext{f(1.)} $\rightarrow$ \cpptext{void f(double)}
\item \cpptext{f("1")} $\rightarrow$ \cpptext{void f(char const*)}
\item \cpptext{f(1.f)} $\rightarrow$  \cpptext{void f(double)}
\item \cpptext{f('1')} $\rightarrow$ \cpptext{void f(int)}
\end{itemize}
}
\only<2>
{
\begin{itemize}
\footnotesize
\item \cpptext{f(1)} $\rightarrow$ \cpptext{void f(int)}
\item \cpptext{f(1.)} $\rightarrow$ \cpptext{void f(double)}
\item \cpptext{f("1")} $\rightarrow$ \cpptext{void f(char const*)}
\item \cpptext{f(1.f)} $\rightarrow$  \cpptext{void f(T)}
\item \cpptext{f('1')} $\rightarrow$ \cpptext{void f(T)}
\end{itemize}
}
\end{block}
\end{frame}

\end{document}
