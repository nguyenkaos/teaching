\documentclass[a4paper]{article}
\usepackage[utf8x]{inputenc}
\usepackage[T1]{fontenc}
\usepackage{fullpage}
\usepackage[french]{babel}
\usepackage{listings}
\usepackage{parskip}

\lstset{language=C++,basicstyle=\small\tt}

\title{\vspace{-2cm}Programmation C++ Avancée}
\author{Joel Falcou \and Guillaume Melquiond}
\date{24 novembre 2015}

\begin{document}
\maketitle

Il sera important de réfléchir à la signature des fonctions à définir. En
particulier, ces signatures devront éviter que des arguments soient
copiés inutilement.

Pour forcer le compilateur à utiliser une version moderne du langage C++,
passez lui l'option \lstinline|--std=c++11| ou l'option
\lstinline|--std=c++14|. Utilisez l'option \lstinline|-Wall| pour activer
tous les diagnostics.

\section{Points}

Définissez un type représentant les points à deux coordonnées flottantes.

Écrivez une fonction qui produit un point avec des coordonnées choisies
au hasard. Note : la fonction \lstinline|random| est déclarée dans
\lstinline|<cstdlib>|.

Écrivez une fonction qui affiche les coordonnées d'un point sur la sortie
standard. Note : les objets et opérateurs
``\lstinline|std::cout << ... << '\n'|'' sont déclarés dans
\lstinline|<iostream>|.

\section{Vecteurs de points}

Le type générique \lstinline|std::vector<T>| fournit des tableaux
dynamiques. Voici quelques opérations supportées :
\begin{itemize}
\item \lstinline|t[i]| permet d'accéder à la case $i$ du tableau $t$;
\item \lstinline|t.size()| renvoie la taille courante du tableau $t$;
\item \lstinline|t.push_back(v)| ajoute une case à la fin du tableau $t$
  et la remplit en utilisant $v$ de type $T$.
\end{itemize}

\smallskip

Écrivez une fonction qui prend un entier $n$ et renvoie un tableau
contenant $n$ points tirés au hasard.

Écrivez une fonction qui affiche le contenu d'un tableau de points.

Remarque : la syntaxe ``\lstinline|for (auto v : t) { ... }|'' itère sur les
cases d'un tableau $t$ et assigne successivement à $v$ les valeurs des
cases.

\section{Point le plus proche}

Écrivez une fonction qui calcule la distance entre deux points. Note :
\lstinline|sqrt| est déclarée dans \lstinline|<cmath>|.

Écrivez une fonction qui, étant donnée un point $p$ et un tableau $t$ de
points, renvoie l'indice dans $t$ du point le plus proche de
$p$.

Utilisez les fonctions définies précédemment pour tester votre code sur
de petits exemples.

\section{Barycentre de points pondérés}

Définissez un type des points pondérés (point muni d'un poids
flottant). Écrivez une fonction qui, étant donnée un tableau de points
pondérés, renvoie le point barycentre de ces points pondérés. Écrivez
les fonctions nécessaires pour tester votre calcul de barycentre.

\end{document}
