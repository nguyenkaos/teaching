% !TEX encoding = UTF-8 Unicode
% -*- coding: UTF-8; -*-
% vim: set fenc=utf-8
\documentclass[svgnames]{beamer}
\setlength{\tabcolsep}{3em}
\usetheme[pageofpages=of,% String used between the current page and the
                         % total page count.
          titleline=true,
          alternativetitlepage=true% Use the fancy title page.
          ]{Torino}
\usecolortheme{metascale}
\usepackage{listings}

\ifxetex
  \usepackage{fontspec}
  \defaultfontfeatures{Mapping=tex-text}
  \setsansfont[ItalicFont={GillSansMTPro-BookItalic}]{GillSansMTPro-Book}
  \setmonofont{Inconsolata}
  \newcommand{\codefont}{\ttfamily}
\else
  \usepackage[utf8x]{inputenc}
  %\usepackage[nott]{inconsolata}
  %\newcommand{\codefont}{\fontfamily{fi4}\selectfont}
  \usepackage{inconsolata}
  \newcommand{\codefont}{\ttfamily}
\fi
\usepackage{graphicx}
\usepackage{color}
%\usepackage{multicol}
%\usepackage{array}
%\usepackage{colortbl}
\usepackage[frenchb]{babel}
%\usepackage{pgfplots}
%\usepackage{standalone}
\usepackage{relsize}

% setup tikz
%% \usepackage{tikz}
%% \usetikzlibrary{calc,trees,positioning,arrows,chains,shapes.geometric,%
%% decorations.pathreplacing,decorations.pathmorphing,shapes,%
%% matrix,shapes.symbols,plotmarks,decorations.markings,shadows,%
%% snakes,backgrounds}
\def\print#1{\pgfmathparse{#1}\pgfmathresult}
\def\colx{Blue!40}
\def\coly{Blue!20}
\def\colz{white}

\def\C++{\textup{C}\nolinebreak[4]\hspace{-.05em}\raisebox{.4ex}{\relsize{-3}{\textbf{++}}}}

% vertical align box
\newcommand*{\vcenteredhbox}[1]{\begingroup
\setbox0=\hbox{#1}\parbox{\wd0}{\box0}\endgroup}

\lstdefinestyle{customcpp}{language=C++,
        basicstyle=\footnotesize\codefont,
        tabsize=2,
        numberstyle=\footnotesize,
        showstringspaces=false,
        %columns=fullflexible,
        keywordstyle=\color[rgb]{0.64,0.13,0.11},
        identifierstyle=,
        commentstyle=\color[rgb]{0.27,0.27,0.28},
        stringstyle=\color[rgb]{0.82,0.16,0.14},
        }

\definecolor{hlcolor}{rgb}{0.88,0.88,0.88}

\NoAutoSpaceBeforeFDP

\title{Programmation C++ Avancée}
\subtitle{Session 2 -- Objets : Modèle et Cycle de vie}
\author{Joel Falcou \and Guillaume Melquiond}
\institute{Laboratoire de Recherche en Informatique}
\date{}

\subject{Computer Science}

\begin{document}
% Copyright © 2014 Lénaïc Bagnères, hnc@singularity.fr

% Licensed under the Apache License, Version 2.0 (the "License");
% you may not use this file except in compliance with the License.
% You may obtain a copy of the License at
%
% http://www.apache.org/licenses/LICENSE-2.0
%
% Unless required by applicable law or agreed to in writing, software
% distributed under the License is distributed on an "AS IS" BASIS,
% WITHOUT WARRANTIES OR CONDITIONS OF ANY KIND, either express or implied.
% See the License for the specific language governing permissions and
% limitations under the License.


% C++
\definecolor{colorkatetext}{RGB}{31,28,27}
\definecolor{colorkatetype}{RGB}{0,87,174}
\definecolor{colorkateint}{RGB}{176,128,0}
\definecolor{colorkatechar}{RGB}{146,76,157}
\definecolor{colorkatestring}{RGB}{191,3,3}
\definecolor{colorkatecomment}{RGB}{137,136,135}

\newcommand{\sourcecodecpp}{\begingroup \catcode`_=12 \sourcecodecppcmd}
\newcommand{\sourcecodecppcmd}[1]{ \lstinputlisting[style=customcpp]{#1} \endgroup}

\newcommand{\sourcecodecpplines}{\begingroup \catcode`_=12 \sourcecodecpplinescmd}
\newcommand{\sourcecodecpplinescmd}[3]{ \lstinputlisting[style=customcpp, firstline=#2, lastline=#3]{#1} \endgroup}

\newcommand{\sourcecodecpplast}{\begingroup \catcode`_=12 \sourcecodecpplastcmd}
\newcommand{\sourcecodecpplastcmd}[2]{ \lstinputlisting[style=customcpp, lastline=#2]{#1} \endgroup}

\newcommand{\sourcecodecppfirst}{\begingroup \catcode`_=12 \sourcecodecppfirstcmd}
\newcommand{\sourcecodecppfirstcmd}[2]{ \lstinputlisting[style=customcpp, firstline=#2]{#1} \endgroup}

\newcommand{\cpptext}[1]{\textcolor{colorkatetext}{\texttt{#1}}}
\newcommand{\cppkeyword}[1]{\textbf{\textcolor{colorkatetext}{\texttt{#1}}}}
\newcommand{\cpptype}[1]{\textcolor{colorkatetype}{\texttt{#1}}}
\newcommand{\cppint}[1]{\textcolor{colorkateint}{\texttt{$#1$}}}
\newcommand{\cppchar}[1]{\textcolor{colorkatechar}{\texttt{#1}}}
\newcommand{\cppstring}[1]{\textcolor{colorkatestring}{\texttt{#1}}}
\newcommand{\cppcomment}[1]{\textcolor{colorkatecomment}{\texttt{#1}}}

\begin{frame}[plain]
\titlepage
\end{frame}
\setcounter{framenumber}{0}

%%%%%%%%%%%%%%%%%% RAPPEL THEORIQUE %%%%%%%%%%%%%%%%%%%%
\section{Qu'est-ce qu'un objet ?}

\frame
{
  \frametitle{Service, Interface, Contrat ?}
  \begin{block}{Un objet - vision logique}
  \begin{itemize}
  \item Un objet encapsule un état
  \item Un objet propose un service
  \item Un objet satisfait à une interface
  \end{itemize}
  \end{block}{}
  
  \begin{block}{Un objet - vision physique}
  \begin{itemize}
  \item un état = données membres
  \item un comportement = fonctions membres
  \item le tout définit dans une \cpptext{class}
  \end{itemize}
  \end{block}{}
}

\frame
{
  \frametitle{Syntaxe de base}
  \sourcecodecpp{code/poo/basic.cpp}  
}
 
\frame
{
  \frametitle{Héritage public}
  \begin{block}{Définition}
  \begin{itemize}
  \footnotesize
  \item Une classe hérite d'une autre afin de spécialiser son comportement
  \item La classe ``fille'' accède aux données et à l'interface publique de sa ``mère''
  \item Notion de sous-classe
  \end{itemize}

  \begin{overlayarea}{\textwidth}{4cm}
  \only<1>{\sourcecodecpplines{code/poo/publicinheritance.cpp}{1}{5}}
  \only<2>{\sourcecodecpplines{code/poo/publicinheritance.cpp}{7}{12}}
  \only<3>{\sourcecodecpplines{code/poo/publicinheritance.cpp}{14}{17}}
  \only<4>{\sourcecodecpplines{code/poo/publicinheritance.cpp}{19}{27}}
  \end{overlayarea}
  \end{block}
}

\frame
{
  \frametitle{Gestion du polymorphisme}
  \sourcecodecpp{code/poo/virtual.cpp}  
}

\frame
{
  \frametitle{Gestion du polymorphisme}
  \sourcecodecpp{code/poo/virtualspec.cpp}  
}

\frame
{
  \frametitle{Principe de substitution de Liskov}
  \begin{block}{\'Enonc\'e}
  Partout o\`u un objet \texttt{x} de type \texttt{T} est attendu, on doit pouvoir passer 
  un objet \texttt{y} de type \texttt{U}, avec \texttt{U}  héritant de \texttt{T} .
  \end{block}

  \begin{block}{Traduction}
  \begin{itemize}
  \item une classe = une interface = un \textbf{contrat}
  \item Les pr\'e-conditions ne peuvent \^etre qu'affaiblies
  \item Les post-conditions ne peuvent \^etre que renforcées
  \end{itemize}
  \end{block}
}

\frame
{
  \frametitle{Principe de substitution de Liskov}
  \only<1>{\sourcecodecpplines{code/poo/liskov.cpp}{3}{14}}
  \only<2>{\sourcecodecpplines{code/poo/liskov.cpp}{16}{30}}
  \only<3>{\sourcecodecpplines{code/poo/liskov.cpp}{32}{49}}
}

\frame
{
  \frametitle{Héritage privé}
  \begin{block}{Définition}
  \begin{itemize}
  \item Résout le problème de la factorisation de code
  \item Permet la réutilisation des composants logiciels
  \item Pas de relation de sous-classe
  \end{itemize}

  \end{block}{}
}

\frame
{
  \frametitle{Héritage privé}
  \sourcecodecpp{code/poo/stack.cpp}
}

\frame
{
  \frametitle{Sémantique de Valeur }
  \begin{block}{Définition}
  Une classe possède une sémantique de valeur ssi deux instances de
  cette classe situées à des adresses différentes, mais au contenu
   identique, sont considérées égales.
  \end{block}

  \begin{block}{Structure classique}
  \begin{itemize}
  \footnotesize
  \item Peut redéfinir des opérateurs ($+$, $-$, $*$, ...)
  \item Possède un opérateur d'affectation
  \item Est comparable via $<$ et $==$
  \item Ne peut être utilisé comme classe de base
  \end{itemize}
  \end{block}
}

\frame
{
 \frametitle{Sémantique d'Entité }
  \begin{block}{Définition}
Une classe a une sémantique d'entité si toutes les instances de
cette classe sont nécessairement deux à deux distinctes. Elle modélise 
un concept d'identité : chaque objet représente un individu unique.
  \end{block}

  \begin{block}{Structure classique}
  \begin{itemize}
  \footnotesize
  \item Ne redéfinit pas d'opérateur
  \item Ne possède pas d'opérateur d'affectation
  \item N'est pas comparable via $<$ et $==$
  \item Les copies sont explicites via une fonction adéquate (clone)
  \end{itemize}
  \end{block}
}

\frame
{
 \frametitle{Que faire ?}
  \begin{block}{Utilisez préférentiellement la sémantique de valeur}
  \begin{itemize}
  \footnotesize
  \item Code clair et concis
  \item Bonne performance car pas d'allocation dynamique supplémentaire
  \item NRVO et élision de copie sont de la partie
  \end{itemize}
  \end{block}

  \begin{block}{Ne délaissez pas la sémantique d'entité}
  \begin{itemize}
  \footnotesize
  \item Extrêmement utile pour le data-driven code
  \item Bonne base pour des bibliothèques d'infrastructure
  \item Se marie élégamment avec les pointeurs à sémantique riche
  \end{itemize}
  \end{block}
}

\section{Cycle de vie en C++}
\frame
{
  \frametitle{Où ranger les objets ?}
  \begin{block}{La pile}
  \begin{itemize}
  \item Mémoire rapide mais limité en espace et en temps
  \item Nettoyage automatique en fin de bloc
  \item Simple et efficace
  \end{itemize}
  \end{block}


  \begin{block}{Le tas}
  \begin{itemize}
  \item Espace virtuellement illimité
  \item Accessible via \cppkeyword{new} et \cppkeyword{delete}
  \item Nécessite une gestion fine
  \end{itemize}
  \end{block}
}

\frame
{
  \frametitle{Principe de RAII}
  \begin{block}{Objectifs}
  \begin{itemize}\footnotesize
  \item Assurer la sûreté de la gestion des ressources
  \item Minimiser la gestion manuelle de la mémoire
  \item Simplifier la gestion des exceptions
  \item Assure une sémantique de valeur
  \end{itemize}
  \end{block}
  
  \begin{center}
  \textbf{R}esource \textbf{A}cquisition \textbf{I}s \textbf{I}nitialisation
  \end{center}

  \begin{block}{Mise en \oe{uvre}}
  \begin{itemize}\footnotesize
  \item Constructeurs = prise de ressource
  \item Destructeur = libération de ressource
  \item Gestion de la ressource au niveau du bloc
  \end{itemize}
  \end{block}
}

\frame
{
  \frametitle{Gestion des Temporaires}
  \begin{block}{\textit{lvalue} vs \textit{rvalue}}
  \begin{itemize}
  \item lvalue : objet avec une identit\'e, un nom
  \item rvalue : objet sans identit\'e
  \item La durée de vie d'une rvalue est en général bornée au \texttt{statement}
  \item Une rvalue peut survivre dans une référence vers une lvalue constante
  \end{itemize}
  \end{block}

  \sourcecodecpp{code/raii/lrvalues.cpp}
}

\frame
{
  \frametitle{R\'ef\'erence vers rvalue}
  \begin{block}{Objectifs}
  \begin{itemize}
  \item Discriminer via un qualificateur lvalue et rvalue
  \item Expliciter les opportunit\'es d'optimisation
  \item Simplifier la d\'efinition d'interfaces
  \end{itemize}
  \end{block}

  \begin{block}{Notation}
  \begin{itemize}
  \item \texttt{T\&} : référence vers lvalue
  \item \texttt{T const\&} : référence vers lvalue constante
  \item \texttt{T\&\&} : référence vers rvalue
  \end{itemize}
  \end{block}
}

\frame
{
  \frametitle{R\'ef\'erence vers rvalue}
  \only<1>
  {
    \begin{block}{Exemple}
    \sourcecodecpp{code/raii/rvalueref.cpp}
    \end{block}
  }
  
  \only<2>
  {
    \begin{block}{Le problème du \texttt{forwarding}}
     \sourcecodecpp{code/raii/forward01.cpp}
    \end{block}
  }
  
  \only<3>
  {
    \begin{block}{Le problème du \texttt{forwarding}}
     \sourcecodecpp{code/raii/forward02.cpp}
    \end{block}
  }
}

\frame
{
  \frametitle{Sémantique de transfert}
  \begin{block}{Problématique}
  \begin{itemize}
  \footnotesize
  \item Copier un objet contenant des ressources est coûteux
  \item Copier depuis un temporaire est doublement coûteux (allocation+deallocation)
  \item Limite l'expressivité de certaines interfaces
  \item Pourquoi ne pas recycler le temporaire ?
  \end{itemize}
  \end{block}

  \begin{block}{Solution}
  \begin{itemize}
  \item Utiliser les rvalue-references pour détecter un temporaire
  \item Extraire son contenu et le \alert{transf\'erer} dans un objet pérenne
  \item Stratégie généralisée à tout le langage et \`a la biblioth\`eque standard
  \end{itemize}
  \end{block}
}

\frame
{
  \frametitle{Sémantique de transfert}
  
  \only<1>{\sourcecodecpplines{code/raii/sort01.cpp}{1}{8}}
  \only<2>{\sourcecodecpplines{code/raii/sort01.cpp}{10}{14}}
  \only<3>{\sourcecodecpplines{code/raii/sort01.cpp}{16}{23}}
  \only<4>{\sourcecodecpplines{code/raii/sort01.cpp}{25}{29}}
}

\end{document}
